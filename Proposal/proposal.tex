\documentclass[a4paper,12pt]{article}
\usepackage[utf8]{inputenc}
\usepackage{geometry}
\usepackage{indentfirst}
\usepackage[colorlinks=true,urlcolor=blue]{hyperref}

\geometry{textheight=27cm} 
\setlength{\parskip}{0.5em}

\title{Machine Learning Engineer Nanodegree \\ Capstone Proposal}
\author{Hao Yun}
\date{\today}

\begin{document}

\maketitle

\section{Domain Background}

Many companies have their reward programs which are designed to encourage people to pay for their products or services. 
Customers are interested in getting rewards. This also benefits companies in attracting and retaining customers. A variety 
of rewards are provided in the market. People may respond differently to certain type of rewards. Comparing with giving the 
whole population the same reward, providing variant rewards at an individual personalized level not only helps companies 
improve the profitability, but also improves customer experience.

When people decide to join a reward program or a membership, companies will get personal information such as name, age, 
gender and financial information. The data generated from the transaction records is equally important to a company.

Analyzing demographic data and transactional data helps companies understand cutsomers better. And contributing a statistical 
model or a machine learning model based on these data can help marketers know customer preferences, evaluate products and 
services, predict trends in the future and make new marketing strategies.

\section{Problem Statement}

Starbucks has provided an experimental dataset on kaggle \cite{kaggle}. Two main problems can be solved by analyzing this 
dataset. One goal is to predict how much a customer will spend at Starbucks during the experiment period. Another is to 
predict the probability that a customer will respond to a reward, so that Starbucks can come up with new strategies to 
determine what kind of reward will be more appropriate for a certain customer.

\section{Datasets and Inputs}

Starbucks dataset contains simulated data that mimics customer behavior on the Starbucks rewards mobile app. And it's a 
simplified version that the simulator has only one product of Starbucks. The dataset has three separate files including offer 
information, demographic data and records of transactions during the test month. More details about the variables of each 
file are available on \href{https://www.kaggle.com/datasets/blacktile/starbucks-app-customer-reward-program-data}{kaggle}.

The objective of this project can be realized by a regression model. Since three separate files are provided, the first step
is to combine the files, then select the features that will be used in the model. 

The target variables are the total amount that each customer spends at Starbucks and the probability that each customer 
completes offers. Not all variables in the dataset are appropriate to predict the target variables. And some variables need 
some processing. Eventually, the dimention of the input data will be 5. The input variables are age, gender, income, the 
number of times customers received offers and the proportion of offers they have viewed.

\section{Solution Statement}

The problem requires the predictions of continuous output variables. This is obviously a supervised learning problem, more 
precisely, a regression problem. The main task is to learn a mapping from multiple input variables to numerical variables. 
Various kinds of algorithms can be used to solve this problem, such as linear regression, random forest and neural 
network. After modeling, output values of an amount and a probability will be predicted from a 5-dimention input data. The 
models can be evaluated by comparing the predicted values and true values.

\section{Benchmark Model}

Since the dataset is not for competition, there isn't a leaderboard or a similar problem on kaggle. So a multivariate linear 
regression model will be used as a benchmark model because it's easy to implement and efficient to train. There are many 
metrics used to evaluate the performance of a regression model like Mean Absolute Error (MAE), Mean Squared Error (MSE), Root 
Mean Squared Error (RMSE) \cite{metric}.

To predict the amount and the probability, there might be other models that perform better with a lower error. MAE measures,  
on average, the absolute values of the differences between predicted values and true values. It will be used to determine 
whether the model is better compared to the linear regression model.

\section{Evaluation Metrics}

Mean absolute error (MAE) is a measure of errors between paired observations expressing the same phenomenon and is calculated 
as the sum of absolute errors divided by the sample size \cite{MAE}:

\begin{equation}
    MAE = \frac{\Sigma_{i=1}^{n}|y_{i} - \hat{y}_{i}|}{n}
\end{equation}

In this project, the main task is to solve a regression problem. MAE is an appropriate metric that can be used to quantify the 
performance of both the benchmark model and the solution model. A low MAE means that predicted values are close to true values.

\section{Project Design}

To implement a regression model, the following steps are required:

\begin{itemize}

    \item Data loading and exploration
    
    Starbucks dataset contains three separate json files. The first step is to combine the variables into a format that can 
    be used in the model which will be contributed based on customer information. Then identify the category of variables, 
    find the distribution of variables and compare correlation of variables.

    \item Data cleaning and pre-processing
    
    There will be missing values and outliers. Typically, outliers can be removed. How to deal with missing values will 
    depend on the situation.

    \item Feature engineering and data transformation
    
    Not all variables are useful to predict the target variables, such as offer id, customer id. And there are variables that 
    can not be used directly in the model like offer type and the time that a transaction happens. Some numerical features can
    be extracted from these variables, for example, the number of times customers received offers or viewed offers.

    Then the processed data will be split into training data and test data. For neural network, data normalisation or 
    standardization is required and very important. However in a linear regression and xgboost, this step is not necessary. In 
    order to compare the metrics of models, data will be normalized or standardized for all models in this project.

    \item Defining and training models, making improvements on the models
    
    Three models will be trained using training data. The benchmark model will be a multivariate linear regression. 
    Additionally, xgboost and neural network will be implemented. 

    After training the model, make predictions on test data. The performance of models can be improved by tuning parameters.
    
    \item Evaluating and comparing model test performance
    
    The final result will be quantified based on metric values. The last step may be to do some reflections according to the 
    process of this project to consider if it exists a better solution or if there are more strategies to improve the model.

\end{itemize}

\begin{thebibliography}{4}

    \bibitem{kaggle}
    \href{https://www.kaggle.com/datasets/blacktile/starbucks-app-customer-reward-program-data}
    {Starbucks app customer rewards program data}

    \bibitem{metric}
    \href{https://stackoverflow.com/questions/60869083/choosing-right-metrics-for-regression-model}
    {Choosing right metrics for regression model}

    \bibitem{MAE}
    \href{https://en.wikipedia.org/wiki/Mean_absolute_error}{Mean absolute error - Wikipedia}

\end{thebibliography}

\end{document}